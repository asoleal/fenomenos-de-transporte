\documentclass[conference]{IEEEtran}
\usepackage{cite}
\usepackage{amsmath,amssymb,amsfonts}
\usepackage{algorithmic}
\usepackage{graphicx}
\usepackage{textcomp}
\usepackage{booktabs}
\usepackage{caption}
\usepackage{subcaption}
\usepackage{siunitx}
\usepackage[spanish]{babel}
\usepackage{hyperref}
\hypersetup{
    colorlinks=true,
    linkcolor=black,
    filecolor=magenta,
    urlcolor=cyan,
}

% --- Definiciones útiles ---
\newcommand{\COtwo}{\ensuremath{\mathrm{CO}_2}}
\newcommand{\CHfour}{\ensuremath{\mathrm{CH}_4}}
\newcommand{\HtwoS}{\ensuremath{\mathrm{H}_2\mathrm{S}}}
\newcommand{\HtwoO}{\ensuremath{\mathrm{H}_2\mathrm{O}}}
\newcommand{\Ntwo}{\ensuremath{\mathrm{N}_2}}

\title{Simulación CFD en Estado Estacionario de las Condiciones de Microclima en el Cultivo de \textit{Pleurotus} spp.: Transporte Acoplado de Aire, Calor, Humedad y \COtwo\ Biogénico}

\author{
    John Jairo Leal Gómez\\
    \textit{Fenómenos de Transporte Avanzados} \\
    Universidad Nacional de Colombia, Palmira\\
    Email: jlealgom@unal.edu.co
}

\begin{document}

\maketitle

\begin{abstract}
Este artículo presenta un modelo de dinámica de fluidos computacional (CFD) en estado estacionario para analizar y optimizar el diseño de sistemas de ventilación en cámaras de cultivo de \textit{Pleurotus ostreatus}, con el objetivo de controlar la acumulación de dióxido de carbono (\COtwo) y mejorar la calidad y rendimiento del hongo. El dominio computacional, de dimensiones $0{,}5 \times 2{,}0 \times 1{,}0$~m, incluye una entrada de aire húmedo ($0{,}2 \times 0{,}2$~m) en la pared superior (xy) y una salida opuesta en la pared xz, posicionadas para favorecer la extracción de gases densos. La emisión de \COtwo, vapor de agua, metano (\CHfour) y sulfuro de hidrógeno (\HtwoS) se modela como fuentes volumétricas constantes en un bloque de sustrato ($0{,}07 \times 1{,}147 \times 0{,}492$~m), sin considerar reacciones químicas ni dinámica metabólica. Mediante ANSYS Fluent, se resuelve acoplado el flujo turbulento (modelo k-$\epsilon$ realizable), la transferencia de calor y el transporte pasivo de especies, con condiciones de frontera físicamente realistas (temperatura uniforme de 20~°C, humedad relativa del 70\%). Los resultados muestran una distribución espacial no uniforme de \COtwo, con acumulación significativa en la capa superior del recinto y una eficiencia de eliminación superior al 95\% mediante la salida. Este enfoque, basado únicamente en principios físicos y sin requerir sensores o datos en tiempo real, ofrece una herramienta predictiva y accesible para el diseño óptimo de ventilación en cultivos controlados de hongos.
\end{abstract}

\begin{IEEEkeywords}
Dinámica de fluidos computacional (CFD), modelado de microclima, cultivo de hongos, transporte de especies, escalares pasivos, \textit{Pleurotus ostreatus}, diseño de ventilación
\end{IEEEkeywords}

\section{Introducción}

La producción comercial de Pleurotus ostreatus depende críticamente del control preciso del microclima dentro de las cámaras de cultivo, donde la concentración de dióxido de carbono (\COtwo), la temperatura y la humedad relativa influyen directamente en la tasa de crecimiento del micelio, la formación de fructificaciones y la calidad final del hongo. En condiciones de cultivo confinado, la respiración aeróbica de los hongos genera \COtwo \ como subproducto metabólico, cuya acumulación por encima de 1000~ppm puede inhibir el desarrollo, reducir el rendimiento y provocar deformaciones morfológicas en los cuerpos fructíferos. A pesar de su importancia, la mayoría de los sistemas de cultivo industriales aún utilizan estrategias de ventilación basadas en temporizadores o sensores de umbral simple (encendido/apagado), que ignoran la dinámica espacial del transporte de gases, los gradientes de concentración y la interacción acoplada entre flujo de aire, transferencia de calor y emisión biogénica.

Este artículo presenta un estudio numérico basado en dinámica de fluidos computacional (CFD) cuyo objetivo principal es determinar el nivel óptimo de ventilación requerido para mantener concentraciones de \COtwo \ por debajo del umbral crítico, maximizando así la producción y calidad del hongo. Para ello, se desarrolla un modelo estacionario que integra de forma acoplada el transporte de especies (\COtwo), vapor de agua, metano y sulfuro de hidrógeno), el flujo turbulento del aire y la transferencia de calor en un dominio tridimensional más o menos realista de dimensiones 0.5~m × 2.0~m × 1.0~m, representativo de una cámara de cultivo típica. La fuente de \COtwo \ se modela como un término volumétrico constante en una región de sustrato fúngico, basado en tasas de emisión reportadas en la literatura, sin considerar reacciones químicas ni dinámica de crecimiento.

Las ecuaciones gobernantes incluyen la conservación de masa, momento y energía, junto con la ecuación de transporte de especies no reactivas, resueltas mediante el modelo de mezcla de gases ideales y el modelo de turbulencia $k-\epsilon$ realizable. La simulación se implementa en ANSYS Fluent versión de prueba, configurando condiciones de frontera físicamente realistas: entrada de aire húmedo con humedad relativa del 70\% y temperatura de 20~°C, salida de presión libre, paredes impermeables y sin deslizamiento, y fuentes volumétricas de especies definidas en la zona del sustrato. La malla se refina en regiones críticas —entrada, salida y fuente— para garantizar convergencia y precisión en los gradientes de concentración.

Inicialmente, éste enfoque evita la dependencia de sensores en tiempo real o modelos de aprendizaje automático, ofrece una herramienta predictiva y de diseño puramente física, capaz de evaluar el desempeño de distintas configuraciones de ventilación antes de su implementación física. Los resultados permiten identificar zonas de acumulación de \COtwo, cuantificar la eficiencia de remoción y proponer mejoras en la ubicación y caudal de las aberturas de entrada y salida, contribuyendo así a la optimización de sistemas de cultivo de hongos bajo condiciones controladas. Por el momento, no hemos parametrizado ninguna de las salidas.


\section{Modelo Matemático y Ecuaciones Gobernantes}

El modelo numérico se basa en la resolución de las ecuaciones de conservación de masa total (continuidad), momento, energía y transporte de masa de especies individuales, para un flujo gaseoso tridimensional, turbulento y en estado estacionario. La ecuación de masa total garantiza el balance global del flujo de aire, mientras que las ecuaciones de transporte de especies describen la distribución espacial de cada componente gaseoso (\COtwo, vapor de agua, metano y sulfuro de hidrógeno) mediante mecanismos de advección y difusión. A continuación se describe cada fenómeno físico modelado, su relevancia en el contexto del cultivo de \textit{Pleurotus ostreatus}, y la formulación matemática correspondiente.

\subsection{Conservación de masa (Continuidad)}

El flujo de aire dentro de la cámara de cultivo debe satisfacer la conservación de la masa total de la mezcla gaseosa. Este principio es fundamental para garantizar que el aire inyectado por la entrada se equilibre con la salida, evitando acumulaciones no físicas de presión o vacíos que alterarían la distribución de gases.

La ecuación de continuidad en estado estacionario es:
\begin{equation}
\nabla \cdot (\rho \mathbf{v}) = 0
\end{equation}
donde $\rho$ es la densidad de la mezcla gaseosa (kg·m$^{-3}$) y $\mathbf{v}$ es el vector velocidad del flujo (m·s$^{-1}$). La densidad se calcula mediante la ley de los gases ideales, acorde con la presión y temperatura del sistema.

%\subsection{Conservación de momento (Flujo de aire turbulento)}

El movimiento del aire es impulsado por la ventilación forzada y modificado por la geometría de la cámara, lo que genera gradientes de velocidad que afectan directamente el transporte de vapor y \COtwo. Dado que los números de Reynolds típicos en cámaras de cultivo superan el régimen laminar, se requiere un modelo de turbulencia para capturar los efectos de mezcla y disipación de energía.

%Se resuelve la ecuación de momento promediada de Reynolds (RANS) en estado estacionario:
%\begin{equation}
%\nabla \cdot (\rho \mathbf{v} \otimes \mathbf{v}) = -\nabla p + \nabla \cdot \left[ \mu_{\text{eff}} \left( \nabla \mathbf{v} + (\nabla \mathbf{v})^\top \right) \right]
%\end{equation}
%donde $p$ es la presión estática (Pa), y $\mu_{\text{eff}} = \mu + \mu_t$ es la viscosidad efectiva, suma de la viscosidad molecular $\mu$ y la viscosidad turbulenta $\mu_t$, esta última obtenida del modelo $k$-$\epsilon$ realizable.

%Este modelo se selecciona por su robustez en flujos con separación moderada y gradientes de presión adversos, comunes cerca de entradas y salidas en cámaras de cultivo.

\subsection{Transferencia de calor}

La temperatura influye tanto en la actividad metabólica del hongo (y por tanto en la tasa de emisión de \COtwo) como en la densidad del aire y la difusión de vapor. Aunque en este estudio se asume una fuente de \COtwo \ constante, el campo térmico debe resolverse porque afecta la convección natural y la humedad relativa.

La ecuación de energía en estado estacionario es:
\begin{equation}
\nabla \cdot (\rho \mathbf{v} h) = \nabla \cdot (k_{\text{eff}} \nabla T)
\end{equation}
donde $h$ es la entalpía específica (J·kg$^{-1}$), $T$ es la temperatura (K), y $k_{\text{eff}} = k + k_t$ es la conductividad térmica efectiva, con $k$ la conductividad molecular y $k_t$ la contribución turbulenta.

Las paredes se mantienen a temperatura constante (20~°C), y no se incluyen fuentes de calor metabólico, ya que la generación térmica por el hongo es despreciable en comparación con el intercambio térmico con las paredes, según estudios en sistemas de cultivo controlado \cite{venturella2019}.

\subsection{Transporte de especies no reactivas}

El \COtwo \ y otros gases emitidos por el sustrato se transportan por advección (arrastrados por el flujo) y difusión molecular en el aire. Esto determina su distribución espacial y su concentración en zonas críticas (micelio, fructificaciones). No se modelan reacciones químicas porque, bajo condiciones aeróbicas normales de cultivo, el \COtwo \ no se consume ni se transforma dentro de la cámara.

Se resuelve, para cada especie $i$, la ecuación de transporte en estado estacionario:
\begin{equation}
\nabla \cdot (\rho \mathbf{v} y_i) = \nabla \cdot (\rho D_{i,\text{eff}} \nabla y_i) + R_i
\end{equation}

donde:

- $y_i$ es la fracción másica de la especie $i$,

- $D_{i,\text{eff}} = D_i + D_{i,t}$ es el coeficiente de difusión efectivo (m$^2$·s$^{-1}$),

- $S_i$ es el término fuente volumétrico (kg·m$^{-3}$·s$^{-1}$), no nulo solo en la región del sustrato.

\subsubsection*{Especies modeladas}

Se consideran cuatro especies gaseosas esenciales en el proceso de respiración aeróbica del hongo y en el equilibrio del microclima, en los valores de $y_i$:
\begin{itemize}
\item \textbf{$O_2$}: reactivo esencial para la respiración celular de \textit{Pleurotus ostreatus}. Su consumo en la zona del sustrato genera un gradiente de concentración que impulsa su difusión desde el aire entrante hacia el micelio. Aunque su concentración en el aire ambiente es alta ($\approx 23\%$), su disminución local puede limitar la tasa metabólica y, por ende, reducir la producción de \COtwo.

    \item \textbf{\COtwo}: producto principal de la respiración aeróbica. Su acumulación en el entorno del micelio, por encima de \SI{1000}{ppm}, inhibe el crecimiento del micelio y la formación de cuerpos fructíferos. Su transporte es dominado por la advección del flujo de aire y la difusión molecular, y se modela como una especie pasiva con fuente volumétrica positiva en la región del sustrato.

    \item \textbf{\HtwoO\ (vapor)}: generado tanto por la liberación metabólica durante la respiración como por la evaporación del agua del sustrato. Su fracción másica influye directamente en la humedad relativa, un parámetro crítico para la turgencia celular, la inducción de primordios y la calidad final del hongo.

    \item \textbf{\Ntwo}: componente mayoritario del aire ($\approx 77\%$), utilizado como especie de referencia para cerrar la composición de la mezcla gaseosa. Su transporte es puramente pasivo, sin fuentes ni sumideros; su fracción másica se determina por cierre de masa:
    \[
    y_{\Ntwo} = 1 - (y_{O_2} + y_{\COtwo} + y_{\HtwoO})
    \]
\end{itemize}
\subsubsection*{Especies y reacciones excluidas}

\vspace{0.5cm}
- \textbf{Oxígeno ($O_2$)}: aunque es consumido en la respiración, su concentración en aire ambiente (~23\%) apenas varía en presencia de fuentes de \COtwo\ modestas. Su disminución es menor al 0.1\%, por lo que no se modela. Esto es consistente con estudios en cámaras bien ventiladas \cite{sanchez2010}.


- \textbf{Reacciones químicas}: No se activan para éste proceso en ANSYS Fluent, pues se supone que no hay reacciones químicas entre las especies.




\subsection{Modelo de turbulencia $k$-$\epsilon$}

El flujo turbulento en la cámara de cultivo se modela mediante el modelo $k$-$\epsilon$ realizable, que resuelve dos ecuaciones adicionales para la energía cinética turbulenta ($k$) y su tasa de disipación ($\epsilon$). Estas ecuaciones, en estado estacionario, se expresan como:

\begin{align}
\nabla \cdot (\rho \mathbf{v} k) &= \nabla \cdot \left( \mu + \frac{\mu_t}{\sigma_k} \right)  + G_k - \rho \epsilon \\
\nabla \cdot (\rho \mathbf{v} \epsilon) &= \nabla \cdot  \left( \mu + \frac{\mu_t}{\sigma_\epsilon} \right) + C_1 \frac{\epsilon}{k} G_k - C_2 \rho \frac{\epsilon^2}{k}
\end{align}

donde:
\begin{itemize}
\item $\rho$: densidad de la mezcla gaseosa (kg·m$^{-3}$)
    \item $\mathbf{v}$: vector velocidad promediada (m·s$^{-1}$)
    \item $k$: energía cinética turbulenta por unidad de masa (m$^2$·s$^{-2}$)
    \item $\epsilon$: tasa de disipación de la energía turbulenta (m$^2$·s$^{-3}$)
    \item $\mu$: viscosidad dinámica molecular del aire (kg·m$^{-1}$·s$^{-1}$)
    \item $\mu_t$: viscosidad turbulenta (kg·m$^{-1}$·s$^{-1}$), calculada como $\mu_t = \rho C_\mu \frac{k^2}{\epsilon}$, con $C_\mu = 0.09$
    \item $\sigma_k$: número de Prandtl turbulento para $k$ (adimensional, valor típico: 1.0)
    \item $\sigma_\epsilon$: número de Prandtl turbulento para $\epsilon$ (adimensional, valor típico: 1.3)
    \item $G_k$: tasa de generación de $k$ por gradientes de velocidad (kg·m$^{-1}$·s$^{-3}$)
    \item $C_1, C_2$: constantes empíricas del modelo ($C_1 = 1.44$, $C_2 = 1.92$ para el modelo estándar)
\end{itemize}
En el modelo \textbf{realizable}, la constante $C_1$ es dinámica y depende del número de deformación $S = \sqrt{2 \mathbf{S}:\mathbf{S}}$, donde $\mathbf{S}$ es el tensor de tasa de deformación:

\[
C_1 = \max\left( 0.43, \frac{\eta}{\eta + 5} \right), \quad \text{con} \quad \eta = \frac{S k}{\epsilon}
\]

Esta variación mejora la predicción en flujos con rotación y curvatura, como los que ocurren cerca de las paredes y en la zona de salida de la cámara.

La viscosidad turbulenta $\mu_t$ se utiliza para cerrar las ecuaciones de Navier-Stokes promediadas (RANS), permitiendo modelar el esfuerzo Reynolds como $\tau_{ij} = \mu_t \left( \frac{\partial u_i}{\partial x_j} + \frac{\partial u_j}{\partial x_i} \right) - \frac{2}{3} \rho k \delta_{ij}$.

Este enfoque es preferido sobre el $k$-$\epsilon$ estándar debido a su mejor desempeño en flujos con rotación y separación, como los que ocurren cerca de las esquinas de la cámara.


\section{Configuración Numérica en ANSYS Fluent}

La simulación se implementa en ANSYS Fluent 2025 R1, utilizando un enfoque estacionario, tridimensional y acoplado para resolver el flujo de aire, la transferencia de calor y el transporte de especies gaseosas en una cámara de cultivo de \textit{Pleurotus ostreatus}. A continuación se describe con detalle la geometría, la malla, los modelos físicos activados, las condiciones de frontera y los criterios de convergencia.

\subsection{Geometría del dominio}

El dominio computacional representa una cámara de cultivo rectangular con dimensiones (ver figura \ref{fig:camara}):
\[
L_x = 0{,}5~\text{m}, \quad L_y = 2{,}0~\text{m}, \quad L_z = 1{,}0~\text{m}.
\]
La malla consta de $54752$ nodos y $276297$ elementos. La entrada de aire $(0.2 x 0.2 m)$ se ubica en la pared superior ($xy$); la salida (idéntica) se coloca en la pared opuesta ($xz$). Esta configuración se parametrizará para evaluar el impacto del diseño de ventilación en la distribución de \COtwo.

\begin{figure}[h!]
    \centering
    \includegraphics[scale=0.5]{camara.png}
    \caption{Cámara de crecimiento de hongos}
    \label{fig:camara}
\end{figure}
El sustrato colonizado por el hongo se modela como una zona activa (no como una fase sólida con flujo, sino como una región o fluido con fuentes volumétricas), con dimensiones:
\[
\Delta x = 0{,}07~\text{m}, \quad \Delta y = 1{,}1~\text{m}, \quad \Delta z = 0{,}492~\text{m},
\]

Se emplea una malla estructurada, generada mediante el módulo \textit{Mesh} de ANSYS Workbench:

La fuente de hongos es una zona porosa, es un sólido, (por el cual se mueve un fluido) con una velocidad resistiva de $100.000m^{-1}$ en cada una de las direcciones.  Con una porosidad de $0.7$. Se asume que la fuente de hongos produce $1.85xe^{-5}kg·m^3·s^{-1}$

\subsection{Modelos físicos activados}

Se activan los siguientes modelos en ANSYS Fluent para capturar los fenómenos relevantes:
\begin{enumerate}

    \item \textbf{Ecuación de Momentum:} Para resolver el campo de velocidad del flujo turbulento.
    
    \item \textbf{Ecuación de energía:} Activada, para resolver el campo térmico acoplado con la densidad del gas ideal.
    
    \item \textbf{Modelo de turbulencia:} $k-\epsilon$ realizable con función de pared estándar. Este modelo es adecuado para flujos internos con separación y recirculación, y ha sido validado en estudios de ventilación agrícola \cite{anderson1995}.
    
    \item \textbf{Transporte de especies:} Activado para una mezcla de cuatro componentes: $O_2$, \COtwo, \HtwoO y \Ntwo. Se utiliza el modelo de difusión de Fick, con coeficientes binarios $D_i$ calculados mediante la correlación de Fuller para mezclas de gases. El consumo de $O_2$ y la producción de \COtwo\ y $H_2O$ se modelan mediante sumideros y fuentes volumétricas prescritas en la zona del sustrato.
    
    \item \textbf{Reacciones químicas:} Desactivadas. Las especies se tratan como escalares pasivos no reactivos, ya que no ocurren transformaciones químicas significativas en el tiempo de residencia del aire (del orden de decenas de segundos). El modelo asume que las emisiones metabólicas son constantes y no dependen de la concentración local de $O_2$ ni de la temperatura.
    
    \item \textbf{Propiedades del material:} La mezcla de gases se define como aire ideal (\textit{ideal-gas mixture}), con densidad variable y propiedades de transporte calculadas por reglas de mezcla estándar.
\end{enumerate}
\subsection{Condiciones de frontera}

Se aplican las siguientes condiciones en las fronteras del dominio:
\begin{itemize}
\item \textbf{Entrada (velocity-inlet):} 
    \begin{itemize}
        \item Velocidad: $0{,}056$~m·s$^{-1}$ (normal a la superficie), lo que corresponde a un caudal volumétrico de $2{,}24 \times 10^{-3}$~m$^3$·s$^{-1}$.
        \item Fracciones másicas: $y_{O_2} = 0.22$, $y_{\COtwo} = 0.006$ (equivalente a aire con  \COtwo\  $\sim$6000~ppm, generado por los hongos ), y $y_{\HtwoO} = 0.01$, equivalente a una humedad relativa del 70\% a 20~°C.
        \item En la entrada se incrementa el vapor de agua para mantener la humedad.
    \end{itemize}
    \item \textbf{Salida (outflow):} Condición de presión libre, asumiendo desarrollo pleno del flujo. En el exterior, se conservan las condiciones ambientales.

    \item \textbf{Paredes (wall):}
    \begin{itemize}
    \item Condición dinámica: sin deslizamiento ($\mathbf{v} = 0$).
        \item Condición térmica: temperatura fija a $20$~°C, consistente con el aire de entrada y representando un entorno térmicamente controlado y uniforme. 
        \item Condición de especies: impermeabilidad total ($-D_i \nabla y_i \cdot \mathbf{n} = 0$), es decir, no hay transferencia de masa a través de las paredes.
    \end{itemize}

\end{itemize}
% \subsection{Términos fuente y solución numérica}

% Los términos fuente volumétricos $S_i$ para cada especie se definen exclusivamente en las celdas que pertenecen al bloque del sustrato, mediante la opción \textit{Cell Zone Conditions} $\rightarrow$ \textit{Source Terms} en Fluent. Los valores empleados se basan en las tasas de emisión reportadas en la Tabla~\ref{tab:sources}.

\subsection{Criterios de convergencia}

La simulación se considera convergida cuando todos los residuos normalizados (continuidad, momento, energía y fracciones másicas) aunque no caen por debajo de $10^{-6}$, debido a los problemas en simulación, se espera que si converjan. Además, se verifica la estabilidad de los balances globales:
\begin{itemize}
\item Balance de masa total: error $< 0.1\%$.
\item Balance de \COtwo: producción = flujo neto por la salida (con error $< 1\%$).
\end{itemize}

\section{Resultados y Discusión}

La simulación en estado estacionario permite caracterizar la distribución espacial de las especies gaseosas, con énfasis en el \COtwo\ como indicador crítico de la calidad del microclima para el cultivo de \textit{Pleurotus ostreatus}. Los resultados revelan patrones de transporte dominados por la ventilación forzada, con efectos secundarios de difusión y ligera convección natural. A continuación se presentan algunos hallazgos iniciales.
\subsection{Convergencia}
A continuación se muestra la gráfica de los residuos, se puede apreciar una buena convergencia para $100$ iteraciones.
\begin{figure}[htbp]
\centering
\includegraphics[width=1.0\linewidth]{remain.png}
\caption{Se muestran los residuos para 100 iteraciones.}
\label{fig:convergencia}
\end{figure}

\subsection{Distribución espacial de \COtwo}

Aunque el sistema de ventilación logra eliminar algo de \COtwo\ en la mayoría del volumen del recinto, la distribución espacial revela que en la zona inmediata al sustrato fúngico —donde se desarrolla el micelio— se acumulan concentraciones elevadas de \COtwo, con fracciones másicas superiores a $2{,}2 \times 10^{-3}$ ($\sim$2200~ppm). Este valor, aunque por debajo del umbral crítico de 5000~ppm, se encuentra dentro del rango en el que se ha reportado inhibición parcial del crecimiento y retraso en la fructificación en cultivos de \textit{Pleurotus ostreatus}. Por tanto, si bien la ventilación es funcional, no es óptima desde el punto de vista fisiológico del hongo, ya que no logra mantener concentraciones uniformes y bajas en la interfaz entre el sustrato y el aire circundante. Esto sugiere que podría requerirse un ajuste en el diseño del flujo —como incrementar el caudal o modificar la ubicación de entrada/salida— para mejorar la calidad del microclima local.
\begin{figure}[htbp]
\centering
\includegraphics[width=1.0\linewidth]{co2.png}
\caption{Gráfico de contornos de la fracción másica de \COtwo\ en un plano vertical central.}
\label{fig:co2_contour}
\end{figure}

\subsection{Distribución de $N_2$}
Se puede apreciar el comportamiento del Nitrógeno
\begin{figure}[htpb]
\centering
\includegraphics[width=1.0\linewidth]{n2.png}
\caption{Gráfico de contornos de la fracción másica de $N_2$ en un plano vertical central.}
\label{fig:n2}
\end{figure}
\subsection{Campo de velocidad}
Se puede apreciar el campo de velocidad, en la entrada, la salida y alrededor de la fuente de hongos.
\begin{figure}[htbp]
\centering
\includegraphics[width=1.0\linewidth]{velocidad.png}
\caption{Gráfico del campo de velocidad.}
\label{fig:velocidad}
\end{figure}

\section{Conclusión}


\begin{thebibliography}{9}
\bibitem{gaitan2004}
Gaitán-Hernández, R., et al., ``Evaluación de sustratos para el cultivo de \textit{Pleurotus ostreatus}'', \emph{Rev. Mex. Mic}, vol. 20, pp. 15–22, 2004.

\bibitem{royse2014}
Royse, D. J., ``Visión general de la industria global de hongos'', en \emph{Hongos Comestibles y Medicinales}, CRC Press, 2014.

\bibitem{sanchez2010}
Sánchez, C., ``Residuos lignocelulósicos: biodegradación y producción de hongos'', \emph{Biotechnol. Adv.}, vol. 28, pp. 1–11, 2010.

\bibitem{venturella2019}
Venturella, F., et al., ``Factores ambientales que afectan el cultivo de hongos'', \emph{Fungal Biol. Rev.}, vol. 33, pp. 1–12, 2019.

\bibitem{cao2021}
Cao, L., et al., ``Compuestos orgánicos volátiles de hongos comestibles: una revisión'', \emph{Food Chem.}, vol. 335, p. 127677, 2021.

\bibitem{ipcc2019}
IPCC, ``Refinamiento 2019 de las Directrices IPCC 2006 para Inventarios Nacionales de Gases de Efecto Invernadero'', 2019.

\bibitem{petrovic2020}
Petrović, M., et al., ``Compuestos odoríferos en granjas de hongos'', \emph{J. Environ. Manage.}, vol. 264, p. 110454, 2020.

\bibitem{fluent2023}
ANSYS, Inc., \emph{Guía de Teoría de ANSYS Fluent}, Edición 2023 R1, 2023.

\bibitem{anderson1995}
Anderson, J. D., \emph{Dinámica de Fluidos Computacional: Los Básicos con Aplicaciones}, McGraw-Hill, 1995.

\end{thebibliography}

\end{document}
