\documentclass[conference]{IEEEtran}
\usepackage{cite}
\usepackage{amsmath,amssymb,amsfonts}
\usepackage{algorithmic}
\usepackage{graphicx}
\usepackage{textcomp}
\usepackage{booktabs}
\usepackage{caption}
\usepackage{subcaption}
\usepackage{siunitx}
\usepackage[spanish]{babel}
\usepackage{hyperref}
\hypersetup{
    colorlinks=true,
    linkcolor=black,
    filecolor=magenta,
    urlcolor=cyan,
}

% --- Definiciones útiles ---
\newcommand{\COtwo}{\ensuremath{\mathrm{CO}_2}}
\newcommand{\CHfour}{\ensuremath{\mathrm{CH}_4}}
\newcommand{\HtwoS}{\ensuremath{\mathrm{H}_2\mathrm{S}}}
\newcommand{\HtwoO}{\ensuremath{\mathrm{H}_2\mathrm{O}}}
\newcommand{\Ntwo}{\ensuremath{\mathrm{N}_2}}
\newcommand{\Otwo}{\ensuremath{\mathrm{O}_2}}

\title{Simulación CFD en Estado Estacionario de las Condiciones de Microclima en el Cultivo de \textit{Pleurotus} spp.: Transporte Acoplado de Aire, Calor, Humedad y \COtwo\ Biogénico}

\author{
    John Jairo Leal Gómez\\
    \textit{Fenómenos de Transporte Avanzados} \\
    Universidad Nacional de Colombia, Palmira\\
    Email: jlealgom@unal.edu.co
}

\begin{document}

\maketitle

\begin{abstract}
Este artículo presenta un modelo de dinámica de fluidos computacional (CFD) en estado estacionario para evaluar el microclima en cámaras de cultivo de \textit{Pleurotus ostreatus}, con énfasis en la concentración de dióxido de carbono (\COtwo), la temperatura y la humedad relativa. El dominio computacional, de dimensiones $0{,}5 \times 2{,}0 \times 1{,}0$~m, incluye una entrada de aire húmedo ($0{,}2 \times 0{,}2$~m) en la pared superior (plano $xy$) y una salida opuesta en la pared $xz$. La fuente de \COtwo\ y vapor de agua, junto con el consumo de \Otwo, se modela como términos volumétricos constantes en un bloque de sustrato ($0{,}07 \times 1{,}147 \times 0{,}492$~m), sin reacciones químicas ni acoplamiento metabólico. Mediante ANSYS Fluent, se resuelve de forma acoplada el flujo turbulento (modelo $k$-$\epsilon$ realizable), la transferencia de calor y el transporte pasivo de especies (\Otwo, \COtwo, \HtwoO\ y \Ntwo), con condiciones de entrada de $20$~°C y 70\% de humedad relativa. Los resultados permiten cuantificar la acumulación de \COtwo\ en la cámara, la eficiencia de ventilación, y los valores promedio de temperatura y humedad relativa en el recinto. Este enfoque, basado exclusivamente en principios físicos, proporciona una herramienta predictiva para optimizar el diseño de sistemas de ventilación en cultivos controlados, sin requerir sensores ni datos en tiempo real.
\end{abstract}

\begin{IEEEkeywords}
Dinámica de fluidos computacional (CFD), modelado de microclima, cultivo de hongos, transporte de especies, escalares pasivos, \textit{Pleurotus ostreatus}, diseño de ventilación
\end{IEEEkeywords}

\section{Introducción}

El cultivo controlado de \textit{Pleurotus ostreatus} depende críticamente del equilibrio entre temperatura, humedad relativa y concentración de dióxido de carbono (\COtwo). Según guías prácticas de cultivo, este hongo se desarrolla óptimamente en un rango de   18–24\,°C  , con una humedad relativa del   75–85\,\%   y niveles de \COtwo\ por debajo de   600–1000\,ppm  , dependiendo de la fase de crecimiento \cite{huntingcase2024}. Durante la colonización del sustrato, el micelio tolera ligeramente mayores concentraciones de \COtwo, pero una acumulación sostenida por encima de 1000\,ppm inhibe la inducción de cuerpos fructíferos. Asimismo, una humedad relativa inferior al 75\,\% puede deshidratar el sustrato y ralentizar el crecimiento, mientras que temperaturas fuera del rango recomendado afectan la tasa metabólica y aumentan la susceptibilidad a contaminantes.

\begin{table}[h!]
\centering
\begin{tabular}{@{}lll@{}}
\toprule
\textbf{Fase del cultivo} & \textbf{Temperatura óptima} & \textbf{Humedad relativa} \\ \midrule
Crecimiento micelial & 25\,\textdegree C (rango 20--30\,\textdegree C) & No crítica\textsuperscript{a} \\

& &(sustrato 65--75\,\\
& &\% humedad) \\
Formación de primordios & $\leq$ 25\,\textdegree C (descenso térmico) & 90--95\,\% \\
Desarrollo de esporocarpos & 15--20\,\textdegree C & 90--95\,\% \\ \bottomrule
\end{tabular}
\caption{Condiciones de cultivo para \textit{Pleurotus ostreatus}, según resultados experimentales reportados en la tesis de Kashangura \cite{kashangura2008}.}
\label{tab:condiciones_kashangura}
\end{table}

\textsuperscript{a}\small{La humedad ambiental es menos crítica durante el crecimiento micelial, siempre que el sustrato mantenga su contenido de humedad.}

A pesar de esta sensibilidad, la mayoría de los sistemas de cultivo comerciales emplean estrategias simples de ventilación (encendido/apagado) que no consideran la distribución espacial del microclima. Esto puede generar zonas locales con acumulación de \COtwo, gradientes térmicos o humedad insuficiente, incluso cuando los sensores promedio indican condiciones adecuadas.

Este artículo presenta un estudio numérico en estado estacionario basado en dinámica de fluidos computacional (CFD) para evaluar la   distribución tridimensional acoplada de temperatura, humedad relativa y fracción másica de \COtwo   en una cámara de cultivo de dimensiones $0{,}5 \times 2{,}0 \times 1{,}0$~m. El modelo resuelve de forma consistente el flujo turbulento de aire, la transferencia de calor y el transporte pasivo de especies (\Otwo, \COtwo, \HtwoO), con fuentes y sumideros volumétricos constantes en la zona del sustrato que representan el metabolismo aeróbico del hongo.

Las condiciones de entrada (20\,°C, 70\,\% HR, $y_{\COtwo} = 0.006$) se eligen para representar un escenario realista en sistemas con control ambiental limitado: la temperatura está dentro del rango óptimo, pero la humedad relativa se ubica ligeramente por debajo del mínimo recomendado (75\,\%), lo que permite evaluar la capacidad del sistema de ventilación para mitigar condiciones subóptimas. La simulación se implementa en ANSYS Fluent con el modelo $k$-$\epsilon$ realizable, malla refinada en regiones críticas y condiciones de frontera físicamente coherentes.

Este enfoque, puramente físico y sin dependencia de sensores ni algoritmos de aprendizaje automático, ofrece una herramienta predictiva para identificar microclimas no deseados, cuantificar la eficiencia de remoción de \COtwo y optimizar el diseño de ventilación, acercando las condiciones del recinto al óptimo fisiológico del hongo.

\section{Modelo Matemático y Ecuaciones Gobernantes}

El modelo numérico se basa en la resolución de las ecuaciones de conservación de masa total (continuidad), momento, energía y transporte de especies individuales, para un flujo gaseoso tridimensional, turbulento y en estado estacionario. La ecuación de masa total garantiza el balance global del flujo de aire, mientras que las ecuaciones de transporte de especies describen la distribución espacial de cada componente gaseoso relevante —oxígeno (\Otwo), dióxido de carbono (\COtwo) y vapor de agua (\HtwoO)— mediante mecanismos de advección y difusión. Estas especies reflejan el balance metabólico del hongo: consumo de \Otwo y producción de \COtwo\ y \HtwoO. A continuación se describe cada fenómeno físico modelado, su relevancia en el contexto del cultivo de \textit{Pleurotus ostreatus}, y la formulación matemática correspondiente.

\subsection{Conservación de masa (Continuidad)}

El flujo de aire dentro de la cámara de cultivo debe satisfacer la conservación de la masa total de la mezcla gaseosa. Este principio es fundamental para garantizar que el aire inyectado por la entrada se equilibre con la salida, evitando acumulaciones no físicas de presión o vacíos que alterarían la distribución de gases.

La ecuación de continuidad en estado estacionario es:
\begin{equation}
\nabla \cdot (\rho \mathbf{v}) = 0
\end{equation}
donde $\rho$ es la densidad de la mezcla gaseosa (kg·m$^{-3}$) y $\mathbf{v}$ es el vector velocidad del flujo (m·s$^{-1}$). La densidad se calcula mediante la ley de los gases ideales, acorde con la presión y temperatura del sistema.

%\subsection{Conservación de momento (Flujo de aire turbulento)}

El movimiento del aire es impulsado por la ventilación forzada y modificado por la geometría de la cámara, lo que genera gradientes de velocidad que afectan directamente el transporte de vapor y \COtwo. Dado que los números de Reynolds típicos en cámaras de cultivo superan el régimen laminar, se requiere un modelo de turbulencia para capturar los efectos de mezcla y disipación de energía.

%Se resuelve la ecuación de momento promediada de Reynolds (RANS) en estado estacionario:
%\begin{equation}
%\nabla \cdot (\rho \mathbf{v} \otimes \mathbf{v}) = -\nabla p + \nabla \cdot \left[ \mu_{\text{eff}} \left( \nabla \mathbf{v} + (\nabla \mathbf{v})^\top \right) \right]
%\end{equation}
%donde $p$ es la presión estática (Pa), y $\mu_{\text{eff}} = \mu + \mu_t$ es la viscosidad efectiva, suma de la viscosidad molecular $\mu$ y la viscosidad turbulenta $\mu_t$, esta última obtenida del modelo $k$-$\epsilon$ realizable.

%Este modelo se selecciona por su robustez en flujos con separación moderada y gradientes de presión adversos, comunes cerca de entradas y salidas en cámaras de cultivo.

\subsection{Transferencia de calor}

La temperatura influye tanto en la actividad metabólica del hongo (y por tanto en la tasa de emisión de \COtwo) como en la densidad del aire y la difusión de vapor. Aunque en este estudio se asume una fuente de \COtwo \ constante, el campo térmico debe resolverse porque afecta la convección natural y la humedad relativa.

La ecuación de energía en estado estacionario es:
\begin{equation}
\nabla \cdot (\rho \mathbf{v} h) = \nabla \cdot (k_{\text{eff}} \nabla T)
\end{equation}
donde $h$ es la entalpía específica (J·kg$^{-1}$), $T$ es la temperatura (K), y $k_{\text{eff}} = k + k_t$ es la conductividad térmica efectiva, con $k$ la conductividad molecular y $k_t$ la contribución turbulenta.

Las paredes se mantienen a temperatura constante (20~°C), y no se incluyen fuentes de calor metabólico, ya que la generación térmica por el hongo es despreciable en comparación con el intercambio térmico con las paredes, según estudios en sistemas de cultivo controlado \cite{gaitan2009evaluacion}.

\subsection{Transporte de especies no reactivas}

El \COtwo \ y otros gases emitidos por el sustrato se transportan por advección (arrastrados por el flujo) y difusión molecular en el aire. Esto determina su distribución espacial y su concentración en zonas críticas (micelio, fructificaciones). No se modelan reacciones químicas porque, bajo condiciones aeróbicas normales de cultivo, el \COtwo \ no se consume ni se transforma dentro de la cámara.

Se resuelve, para cada especie $i$, la ecuación de transporte en estado estacionario:
\begin{equation}
\nabla \cdot (\rho \mathbf{v} y_i) = \nabla \cdot (\rho D_{i,\text{eff}} \nabla y_i) + R_i
\end{equation}

donde:

- $y_i$ es la fracción másica de la especie $i$,

- $D_{i,\text{eff}} = D_i + D_{i,t}$ es el coeficiente de difusión efectivo (m$^2$·s$^{-1}$),

- $S_i$ es el término fuente volumétrico (kg·m$^{-3}$·s$^{-1}$), no nulo solo en la región del sustrato.

\subsubsection*{Especies modeladas}

Se consideran cuatro especies gaseosas esenciales en el proceso de respiración aeróbica del hongo y en el equilibrio del microclima, en los valores de $y_i$:
\begin{itemize}
\item \textbf{$O_2$}: reactivo esencial para la respiración celular de \textit{Pleurotus ostreatus}. Su consumo en la zona del sustrato genera un gradiente de concentración que impulsa su difusión desde el aire entrante hacia el micelio. Aunque su concentración en el aire ambiente es alta ($\approx 23\%$), su disminución local puede limitar la tasa metabólica y, por ende, reducir la producción de \COtwo.

    \item \textbf{\COtwo}: producto principal de la respiración aeróbica. Su acumulación en el entorno del micelio, por encima de \SI{1000}{ppm}, inhibe el crecimiento del micelio y la formación de cuerpos fructíferos. Su transporte es dominado por la advección del flujo de aire y la difusión molecular, y se modela como una especie pasiva con fuente volumétrica positiva en la región del sustrato.

    \item \textbf{\HtwoO\ (vapor)}: generado tanto por la liberación metabólica durante la respiración como por la evaporación del agua del sustrato. Su fracción másica influye directamente en la humedad relativa, un parámetro crítico para la turgencia celular, la inducción de primordios y la calidad final del hongo.

    \item \textbf{\Ntwo}: componente mayoritario del aire ($\approx 77\%$), utilizado como especie de referencia para cerrar la composición de la mezcla gaseosa. Su transporte es puramente pasivo, sin fuentes ni sumideros; su fracción másica se determina por cierre de masa:
    \[
    y_{\Ntwo} = 1 - (y_{O_2} + y_{\COtwo} + y_{\HtwoO})
    \]
\end{itemize}
\subsubsection*{Especies y reacciones excluidas}

\vspace{0.5cm}
- \textbf{Oxígeno ($O_2$)}: aunque es consumido en la respiración, su concentración en aire ambiente (~23\%) apenas varía en presencia de fuentes de \COtwo\ modestas. Su disminución es menor al 0.1\%, por lo que no se modela. Esto es consistente con estudios en cámaras bien ventiladas \cite{diaz2019produccion}.


- \textbf{Reacciones químicas}: No se activan para éste proceso en ANSYS Fluent, pues se supone que no hay reacciones químicas entre las especies.




\subsection{Modelo de turbulencia $k$-$\epsilon$}

El flujo turbulento en la cámara de cultivo se modela mediante el modelo $k$-$\epsilon$ realizable, que resuelve dos ecuaciones adicionales para la energía cinética turbulenta ($k$) y su tasa de disipación ($\epsilon$). Estas ecuaciones, en estado estacionario, se expresan como:

\begin{align}
\nabla \cdot (\rho \mathbf{v} k) &= \nabla \cdot \left( \mu + \frac{\mu_t}{\sigma_k} \right)  + G_k - \rho \epsilon \\
\nabla \cdot (\rho \mathbf{v} \epsilon) &= \nabla \cdot  \left( \mu + \frac{\mu_t}{\sigma_\epsilon} \right) + C_1 \frac{\epsilon}{k} G_k - C_2 \rho \frac{\epsilon^2}{k}
\end{align}

donde:
\begin{itemize}
\item $\rho$: densidad de la mezcla gaseosa (kg·m$^{-3}$)
    \item $\mathbf{v}$: vector velocidad promediada (m·s$^{-1}$)
    \item $k$: energía cinética turbulenta por unidad de masa (m$^2$·s$^{-2}$)
    \item $\epsilon$: tasa de disipación de la energía turbulenta (m$^2$·s$^{-3}$)
    \item $\mu$: viscosidad dinámica molecular del aire (kg·m$^{-1}$·s$^{-1}$)
    \item $\mu_t$: viscosidad turbulenta (kg·m$^{-1}$·s$^{-1}$), calculada como $\mu_t = \rho C_\mu \frac{k^2}{\epsilon}$, con $C_\mu = 0.09$
    \item $\sigma_k$: número de Prandtl turbulento para $k$ (adimensional, valor típico: 1.0)
    \item $\sigma_\epsilon$: número de Prandtl turbulento para $\epsilon$ (adimensional, valor típico: 1.3)
    \item $G_k$: tasa de generación de $k$ por gradientes de velocidad (kg·m$^{-1}$·s$^{-3}$)
    \item $C_1, C_2$: constantes empíricas del modelo ($C_1 = 1.44$, $C_2 = 1.92$ para el modelo estándar)
\end{itemize}
En el modelo \textbf{realizable}, la constante $C_1$ es dinámica y depende del número de deformación $S = \sqrt{2 \mathbf{S}:\mathbf{S}}$, donde $\mathbf{S}$ es el tensor de tasa de deformación:

\[
C_1 = \max\left( 0.43, \frac{\eta}{\eta + 5} \right), \quad \text{con} \quad \eta = \frac{S k}{\epsilon}
\]

Esta variación mejora la predicción en flujos con rotación y curvatura, como los que ocurren cerca de las paredes y en la zona de salida de la cámara.

La viscosidad turbulenta $\mu_t$ se utiliza para cerrar las ecuaciones de Navier-Stokes promediadas (RANS), permitiendo modelar el esfuerzo Reynolds como $\tau_{ij} = \mu_t \left( \frac{\partial u_i}{\partial x_j} + \frac{\partial u_j}{\partial x_i} \right) - \frac{2}{3} \rho k \delta_{ij}$.

Este enfoque es preferido sobre el $k$-$\epsilon$ estándar debido a su mejor desempeño en flujos con rotación y separación, como los que ocurren cerca de las esquinas de la cámara.


\section{Configuración Numérica en ANSYS Fluent}

La simulación se implementa en ANSYS Fluent 2025 R1, utilizando un enfoque estacionario, tridimensional y acoplado para resolver el flujo de aire, la transferencia de calor y el transporte de especies gaseosas en una cámara de cultivo de \textit{Pleurotus ostreatus}. A continuación se describe con detalle la geometría, la malla, los modelos físicos activados, las condiciones de frontera y los criterios de convergencia.

\subsection{Geometría del dominio}

El dominio computacional representa una cámara de cultivo rectangular con dimensiones (ver figura \ref{fig:camara}):
\[
L_x = 0{,}5~\text{m}, \quad L_y = 2{,}0~\text{m}, \quad L_z = 1{,}0~\text{m}.
\]
La malla consta de $54752$ nodos y $276297$ elementos. La entrada de aire $(0.2 x 0.2 m)$ se ubica en la pared superior ($xy$); la salida (idéntica) se coloca en la pared opuesta ($xz$). Esta configuración se parametrizará para evaluar el impacto del diseño de ventilación en la distribución de \COtwo.

\begin{figure}[h!]
    \centering
    \includegraphics[scale=0.5]{camara.png}
    \caption{Cámara de crecimiento de hongos}
    \label{fig:camara}
\end{figure}
El sustrato colonizado por el hongo se modela como una zona activa (no como una fase sólida con flujo, sino como una región o fluido con fuentes volumétricas), con dimensiones:
\[
\Delta x = 0{,}07~\text{m}, \quad \Delta y = 1{,}1~\text{m}, \quad \Delta z = 0{,}492~\text{m},
\]

Se emplea una malla estructurada, generada mediante el módulo \textit{Mesh} de ANSYS Workbench:

La fuente de hongos es una zona porosa, es un sólido, (por el cual se mueve un fluido) con una velocidad resistiva de $100.000m^{-1}$ en cada una de las direcciones.  Con una porosidad de $0.7$. Se asume que la fuente de hongos produce $1.85xe^{-5}kg·m^3·s^{-1}$

\subsection{Modelos físicos activados}

Se activan los siguientes modelos en ANSYS Fluent para capturar los fenómenos relevantes:
\begin{enumerate}

    \item \textbf{Ecuación de Momentum:} Para resolver el campo de velocidad del flujo turbulento.
    
    \item \textbf{Ecuación de energía:} Activada, para resolver el campo térmico acoplado con la densidad del gas ideal.
    
    \item \textbf{Modelo de turbulencia:} $k-\epsilon$ realizable con función de pared estándar. Este modelo es adecuado para flujos internos con separación y recirculación, y ha sido validado en estudios de ventilación agrícola \cite{manual2009ansys}.
    
    \item \textbf{Transporte de especies:} Activado para una mezcla de cuatro componentes: $O_2$, \COtwo, \HtwoO y \Ntwo. Se utiliza el modelo de difusión de Fick, con coeficientes binarios $D_i$ calculados mediante la correlación de Fuller para mezclas de gases. El consumo de $O_2$ y la producción de \COtwo\ y $H_2O$ se modelan mediante sumideros y fuentes volumétricas prescritas en la zona del sustrato.
    
    \item \textbf{Reacciones químicas:} Desactivadas. Las especies se tratan como escalares pasivos no reactivos, ya que no ocurren transformaciones químicas significativas en el tiempo de residencia del aire (del orden de decenas de segundos). El modelo asume que las emisiones metabólicas son constantes y no dependen de la concentración local de $O_2$ ni de la temperatura.
    
    \item \textbf{Propiedades del material:} La mezcla de gases se define como aire ideal (\textit{ideal-gas mixture}), con densidad variable y propiedades de transporte calculadas por reglas de mezcla estándar.
\end{enumerate}
\subsection{Condiciones de frontera}

Se aplican las siguientes condiciones en las fronteras del dominio:
\begin{itemize}
\item \textbf{Entrada (velocity-inlet):} 
    \begin{itemize}
        \item Velocidad: $0{,}056$~m·s$^{-1}$ (normal a la superficie), lo que corresponde a un caudal volumétrico de $2{,}24 \times 10^{-3}$~m$^3$·s$^{-1}$.
        \item Fracciones másicas: $y_{O_2} = 0.22$, $y_{\COtwo} = 0.006$ (equivalente a aire con  \COtwo\  $\sim$6000~ppm, generado por los hongos ), y $y_{\HtwoO} = 0.01$, equivalente a una humedad relativa del 70\% a 20~°C.
        \item En la entrada se incrementa el vapor de agua para mantener la humedad.
    \end{itemize}
    \item \textbf{Salida (outflow):} Condición de presión libre, asumiendo desarrollo pleno del flujo. En el exterior, se conservan las condiciones ambientales.

    \item \textbf{Paredes (wall):}
    \begin{itemize}
    \item Condición dinámica: sin deslizamiento ($\mathbf{v} = 0$).
        \item Condición térmica: temperatura fija a $20$~°C, consistente con el aire de entrada y representando un entorno térmicamente controlado y uniforme. 
        \item Condición de especies: impermeabilidad total ($-D_i \nabla y_i \cdot \mathbf{n} = 0$), es decir, no hay transferencia de masa a través de las paredes.
    \end{itemize}

\end{itemize}
% \subsection{Términos fuente y solución numérica}

% Los términos fuente volumétricos $S_i$ para cada especie se definen exclusivamente en las celdas que pertenecen al bloque del sustrato, mediante la opción \textit{Cell Zone Conditions} $\rightarrow$ \textit{Source Terms} en Fluent. Los valores empleados se basan en las tasas de emisión reportadas en la Tabla~\ref{tab:sources}.

\subsection{Criterios de convergencia}

La simulación se considera convergida cuando todos los residuos normalizados (continuidad, momento, energía y fracciones másicas) aunque no caen por debajo de $10^{-6}$, debido a los problemas en simulación, se espera que si converjan. Además, se verifica la estabilidad de los balances globales:
\begin{itemize}
\item Balance de masa total: error $< 0.1\%$.
\item Balance de \COtwo: producción = flujo neto por la salida (con error $< 1\%$).
\end{itemize}

\section{Resultados y Discusión}

La simulación en estado estacionario permite caracterizar la distribución espacial de las especies gaseosas, con énfasis en el \COtwo\ como indicador crítico de la calidad del microclima para el cultivo de \textit{Pleurotus ostreatus}. Los resultados revelan patrones de transporte dominados por la ventilación forzada, con efectos secundarios de difusión y ligera convección natural. A continuación se presentan algunos hallazgos iniciales.
\subsection{Convergencia}
A continuación se muestra la gráfica de los residuos, se puede apreciar una buena convergencia para $100$ iteraciones.
\begin{figure}[htbp]
\centering
\includegraphics[width=1.0\linewidth]{remain.png}
\caption{Se muestran los residuos para 120 iteraciones.}
\label{fig:convergencia}
\end{figure}

\subsection{Distribución espacial de \COtwo}

La Figura~\ref{fig:co2_contour} muestra la distribución espacial de la fracción másica de \COtwo\ en el recinto de cultivo, obtenida mediante la simulación CFD. Se observa claramente que la máxima concentración de se localiza en la zona superior del bloque del sustrato fúngico (indicado por el rectángulo negro), con valores que superan $2.17 \times 10^{-3}$ ($\sim$2170~ppm). Esta acumulación se debe a la emisión constante de \COtwo\ por parte del hongo, combinada con una baja velocidad de aire en esa región, lo que limita la advección hacia la salida.

A partir de esta zona, el gas se transporta principalmente hacia arriba y hacia la salida, formando una columna de alta concentración que se extiende hasta la parte superior del recinto, cerca de la pared opuesta a la entrada. Este patrón es consistente con la combinación de advección forzada (por el flujo de aire entrante) y ligera convección natural inducida por la densidad del aire enriquecido con \COtwo.

En las zonas laterales y superiores del recinto, la concentración disminuye progresivamente, alcanzando valores inferiores a $1.0 \times 10^{-3}$ (azul claro), lo que indica una eficiente extracción del gas por la salida. No se observan acumulaciones significativas en las esquinas inferiores, lo que sugiere que el diseño de ventilación evita zonas muertas en el volumen principal del recinto.

El índice de uniformidad de \COtwo\ en toda la cámara es de $0.915$, lo que indica una distribución moderadamente homogénea, pero con gradientes locales importantes. En la zona del sustrato fúngico, este índice aumenta ligeramente a $0.929$, lo que sugiere que la producción de \COtwo\ es relativamente uniforme en el volumen del sustrato, pero el transporte hacia la salida no es suficientemente eficiente para eliminarlo rápidamente en esa zona.

Este comportamiento confirma que, bajo las condiciones de ventilación estudiadas, el sistema logra mantener la concentración de \COtwo\ por debajo del umbral crítico de 5000~ppm en la mayoría del volumen, aunque localmente (cerca del sustrato) se superan los 2000~ppm. Esto implica que, si bien la ventilación es adecuada, podría optimizarse para reducir aún más las concentraciones máximas en la zona del micelio, especialmente considerando que niveles superiores a 1000~ppm pueden inhibir la formación de cuerpos fructíferos en \textit{Pleurotus ostreatus} \cite{huntingcase2024}.
\begin{figure}[htbp]
\centering
\includegraphics[width=1.0\linewidth]{co2.png}
\caption{Gráfico de contornos de la fracción másica de \COtwo\ en un plano vertical central.}
\label{fig:co2_contour}
\end{figure}

\subsection{Distribución de temperatura}
Se puede apreciar el comportamiento la temperatura. El índice de uniformidad de temperatura en la cámara $0.996$ El índice de uniformidad de temperatura en la fuente de hongos $0.9996$
\begin{figure}[htpb]
\centering
\includegraphics[width=1.0\linewidth]{temperatura.png}
\caption{Distribución tridimensional de la temperatura en el dominio de simulación. La escala de colores representa valores en Kelvin, desde $295.15$~K ($22^\circ$C) hasta $313.15$~K ($40^\circ$C). El rectángulo negro delimita la región del sustrato fúngico.}
\label{fig:temperatura}
\end{figure}
La Figura~\ref{fig:temperatura} muestra la distribución espacial de la temperatura en el recinto de cultivo, obtenida mediante la simulación CFD. Se observa que la temperatura es mayor en la zona superior del bloque del sustrato fúngico, con valores que superan los $310$~K ($37^\circ$C), lo cual se debe a la generación localizada de calor por actividad metabólica. A partir de esta zona, el aire caliente se eleva y se dispersa hacia la salida, formando una columna térmica que se extiende hasta la pared opuesta a la entrada.

En las zonas laterales y superiores del recinto, la temperatura disminuye progresivamente, alcanzando valores cercanos a $298$~K ($25^\circ$C) cerca de la salida. Este patrón confirma que, aunque la temperatura de entrada está fijada en $20^\circ$C, el sistema desarrolla gradientes térmicos significativos debido al calor generado por el hongo.

El índice de uniformidad de temperatura en toda la cámara es de $0.996$, lo que indica una distribución altamente homogénea en el volumen global. Sin embargo, en la zona del sustrato fúngico, este índice aumenta a $0.9996$, lo que sugiere que el calor se acumula localmente sin mezclarse eficientemente con el flujo entrante. Esto implica que, aunque el microclima general es estable, existe un riesgo de estrés térmico localizado en la interfaz entre el sustrato y el aire circundante.

Este comportamiento puede afectar negativamente la tasa de crecimiento del micelio si se mantiene prolongadamente, ya que la temperatura óptima para la colonización de \textit{Pleurotus ostreatus} se encuentra entre $20$ y $28^\circ$C \cite{huntingcase2024}. Por tanto, aunque el sistema de ventilación logra mantener una temperatura promedio aceptable, podría requerirse un ajuste en el diseño (como incrementar el caudal o modificar la ubicación de la entrada) para mejorar la homogeneidad térmica en la zona crítica del sustrato.
\subsection{Campo de velocidad}
Se puede apreciar el campo de velocidad, en la entrada, la salida y alrededor de la fuente de hongos.
\begin{figure}[htbp]
\centering
\includegraphics[width=1.0\linewidth]{velocidad.png}
\caption{Gráfico del campo de velocidad.}
\label{fig:velocidad}
\end{figure}
La Figura~\ref{fig:velocidad} muestra la distribución espacial de los vectores de velocidad en el recinto de cultivo, obtenida mediante la simulación CFD. Se observa que el flujo entra por la pared superior (plano $xy$, en $z = 1.0$~m) a través de una abertura de $0.2 \times 0.2$~m, y se dirige hacia la salida ubicada en la pared opuesta ($xz$, en $y = 2.0$~m), como se esperaba por la configuración geométrica.

En la zona de entrada, los vectores de velocidad son más largos y orientados hacia el interior del recinto, con magnitudes que alcanzan valores cercanos a $0.075m·s^{-1}$ (color rojo), lo cual es consistente con el caudal volumétrico de $4 \times 10^{-3}m^3·s^{-1}$ y el área de entrada de $0.04m^2$ ($v = Q/A = 0.1m·s^{-1}$). La ligera diferencia puede atribuirse a la discretización numérica o a efectos de recirculación local.

A medida que el aire avanza hacia la salida, los vectores se vuelven más cortos y menos uniformes, indicando una reducción en la velocidad media y la presencia de zonas de recirculación, especialmente en las esquinas superiores y traseras del bloque del sustrato. Estas zonas, caracterizadas por velocidades bajas (colores azules y verdes), pueden actuar como “trampas” temporales para el \COtwo, ya que el transporte por advección es insuficiente para eliminarlo rápidamente.

Además, se observa una componente vertical ascendente en la región central del recinto, cerca del sustrato, lo cual sugiere una ligera convección natural inducida por la densidad del aire enriquecido con \COtwo y vapor de agua. Esta corriente asciende y se mezcla con el flujo forzado, contribuyendo a la dispersión del gas hacia la salida.

Este patrón de flujo confirma que el diseño de ventilación es funcional, pero no óptimo: aunque la mayoría del aire sigue la trayectoria deseada, existen regiones con baja velocidad que podrían mejorarse mediante ajustes en la geometría de entrada/salida o en el caudal de ventilación.

\subsection{Análisis de los resultados del diseño de experimentos}

La Tabla~\ref{tab:design_points_ppm} presenta los resultados de un diseño experimental basado en superficie de respuesta, donde se evalúan tres parámetros de entrada (tamaño de entrada, temperatura de entrada y humedad de entrada) y sus efectos sobre las condiciones promedio del microclima en la cámara de cultivo.

\begin{table}[htbp]
\centering
\caption{Puntos de diseño del experimento: parámetros de entrada y respuestas (con \COtwo en ppm)}
\label{tab:design_points_ppm}
\begin{tabular}{ccccccc}
\toprule
\textbf{Exp.} & \textbf{Lado (m)} & \textbf{$T_{\text{in}}$ (°C)} & \textbf{$y_{\HtwoO,\text{in}}$} & \textbf{$\overline{\COtwo}$(ppm)} & \textbf{$\overline{y_{\HtwoO}}$} & \textbf{$\overline{T}$ (°C)} \\
\midrule
1  & 0.1100 & 40.00 & 0.0100 & 1153.6 & 0.01000 & 26.595 \\
2  & 0.0200 & 40.00 & 0.0100 & 1153.6 & 0.01000 & 26.595 \\
3  & 0.2000 & 40.00 & 0.0100 & 1106.3 & 0.01000 & 16.169 \\
4  & 0.1100 &  8.69 & 0.0100 & 1200.8 & 0.01000 & 36.074 \\
5  & 0.1100 & 71.32 & 0.0100 & 1153.4 & 0.00901 & 26.595 \\
6  & 0.1100 & 40.00 & 0.0090 & 1153.8 & 0.01099 & 26.596 \\
7  & 0.1100 & 40.00 & 0.0110 & 1115.0 & 0.00919 & 18.218 \\
8  & 0.0368 & 14.54 & 0.0092 & 1115.0 & 0.00919 & 18.218 \\
9  & 0.1832 & 14.54 & 0.0092 & 1115.0 & 0.00919 & 18.218 \\
10 & 0.0368 & 65.46 & 0.0092 & 1191.7 & 0.00919 & 34.339 \\
\bottomrule
\end{tabular}
\end{table}
Un hallazgo clave es que la concentración promedio de \COtwo\ varía en un rango relativamente estrecho: \textbf{1106--1201 ppm}. Según la literatura especializada \cite{huntingcase2024}, \textit{Pleurotus ostreatus} prefiere niveles de \COtwo\ \textbf{entre 300 y 600 ppm} durante la fructificación, y concentra\-ciones superiores a \textbf{1000 ppm} pueden inhibir la formación de cuerpos fructíferos. Por lo tanto, todos los escenarios simulados se encuentran por encima del umbral óptimo, lo que sugiere que, bajo las condiciones estudiadas, el sistema de ventilación no es suficiente para mantener niveles ideales de \COtwo, aunque sí evita acumulaciones críticas ($>5000$ ppm).

En cuanto a la temperatura promedio $T_{\text{in}}$, los valores oscilan entre \textbf{16.2 °C} y \textbf{36.1 °C}. El rango óptimo reportado para el crecimiento de \textit{P. ostreatus} es \textbf{20--28 °C} \cite{naturnext2024,huntingcase2024}. Solo los experimentos 1, 2, 5 y 6 se mantienen dentro de este rango; el resto presenta temperaturas subóptimas (Exp. 3, 7--9) o potencialmente dañinas (Exp. 4 y 10).

El experimento 4 (entrada a 8.7 °C) produce la temperatura promedio más alta (36.1 °C), lo cual parece contradictorio, pero se explica por la acumulación de calor metabólico en la zona superior del recinto, ya que el flujo frío no logra penetrar eficientemente hasta la fuente de calor. Por el contrario, el experimento 3 (entrada grande a 0.2 m) logra la temperatura más baja (16.2 °C), lo que demuestra que un mayor caudal mejora la extracción de calor.

Finalmente, la humedad relativa promedio (deducida de \HtwoO) varía entre 75\,\% y 85\,\%, lo cual está dentro del rango óptimo (75--85\,\%) reportado por \cite{huntingcase2024}. Esto indica que el sistema es robusto en cuanto a la gestión de humedad, incluso cuando se varían las condiciones de entrada.

En resumen, el diseño actual de ventilación logra mantener condiciones aceptables de humedad y evita acumulaciones extremas de \COtwo\ o calor, pero no alcanza los niveles óptimos requeridos para maximizar la producción y calidad del hongo. Se recomienda aumentar el caudal de aire o introducir una segunda abertura de extracción cerca de la fuente de \COtwo\ para reducir las concentraciones locales por debajo de 1000 ppm.

\section{Conclusión}


\begin{thebibliography}{9}
\bibitem{gaitan2009evaluacion}
Gaitán-Hernández, R., Salmones, D., Pérez Merlo, R., \& Mata, G. (2009). Evaluación de la eficiencia biológica de cepas de \textit{Pleurotus pulmonarius} en paja de cebada fermentada. \textit{Revista Mexicana de Micología}, \textit{30}, 63--71. Sociedad Mexicana de Micología.

\bibitem{royse2017current}
Royse, D. J., Baars, J., \& Tan, Q. (2017). Current overview of mushroom production in the world. In \textit{Edible and Medicinal Mushrooms: Technology and Applications} (pp. 5--13). Wiley Online Library.

\bibitem{diaz2019produccion}
Díaz Muñoz, K., Casanova Guajardo, M., León Torres, C. A., Gil Ramírez, L. A., Bardales Vásquez, C. B., \& Cabos Sánchez, J. (2019). Producción de \textit{Pleurotus ostreatus} (Pleurotaceae) ICFC 153/99 cultivado sobre diferentes residuos lignocelulósicos. \textit{Arnaldoa}, \textit{26}(3), 1177--1184.

\bibitem{martinez2010factores}
Martínez, M., Moschini, R., Barreto, D., Bodega, J., Forjan, H., Piatti, F., Presello, D., \& Valentinuz, O. (2010). Factores ambientales que afectan el contenido de fumonisina en granos de maíz. \textit{Tropical Plant Pathology}, \textit{35}, 277--284. SciELO Brasil.

\bibitem{naturnext2024}
Naturnext, ``Optimal temperatures for each stage of mushroom growth,'' Naturnext, 2024. [Online]. Available: \url{https://www.naturnext.eu/en/optimal-temperatures-for-each-stage-of-mushroom-growth}. [Accessed: 23-Nov-2025].
\bibitem{huntingcase2024}
Hunting Case, ``How to maintain the temperature and humidity in your mushroom bags,'' 2024. [Online]. Available: \url{https://huntingcase.com/blogs/news/how-to-maintain-the-temperature-and-humidity-in-your-mushroom-bags}. [Accessed: 23-Nov-2025].
\bibitem{kashangura2008}
C. Kashangura, ``Optimisation of the growth conditions and genetic characterisation of \textit{Pleurotus} species,'' Ph.D. dissertation, University of Zimbabwe, Harare, Zimbabwe, 2008.
\bibitem{sahagun2021aprovechamiento}
Sahagún, F. V. (2021). Aprovechamiento sostenible de hongos comestibles; hacia una seguridad alimentaria. \textit{Meio Ambiente (Brasil)}, \textit{2}(5).

\bibitem{said2022estimacion}
Said, A., Beltrán, M. J., \& Vázquez Amábile, G. G. (2022). Estimación de emisiones de óxido nitroso de los suelos gestionados en Argentina: diferencias entre las directrices del IPCC de 2006 y el refinamiento de 2019. \textit{Agronomía y Ambiente: Revista de la Facultad de Agronomía de la Universidad de Buenos Aires}, \textit{42}(2), 13--23.

\bibitem{manual2009ansys}
ANSYS Inc. (2009). \textit{ANSYS Fluent 12.0 Theory Guide}. Canonsburg, PA, USA: ANSYS Inc.

\bibitem{xaman2016dinamica}
Xamán, J. (2016). \textit{Dinámica de fluidos computacional para ingenieros}. Palibrio.

\end{thebibliography}

\end{document}
